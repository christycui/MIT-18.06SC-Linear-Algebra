%=======================02-713 LaTeX template, following the 15-210 template==================
%
% You don't need to use LaTeX or this template, but you must turn your homework in as
% a typeset PDF somehow.
%
% How to use:
%    1. Update your information in section "A" below
%    2. Write your answers in section "B" below. Precede answers for all 
%       parts of a question with the command "\question{n}{desc}" where n is
%       the question number and "desc" is a short, one-line description of 
%       the problem. There is no need to restate the problem.
%    3. If a question has multiple parts, precede the answer to part x with the
%       command "\part{x}".
%    4. If a problem asks you to design an algorithm, use the commands
%       \algorithm, \correctness, \runtime to precede your discussion of the 
%       description of the algorithm, its correctness, and its running time, respectively.
%    5. You can include graphics by using the command \includegraphics{FILENAME}
%
\documentclass[11pt]{article}
\usepackage{amsmath,amssymb,amsthm}
\usepackage{graphicx}
\usepackage[margin=1in]{geometry}
\usepackage{fancyhdr}
\setlength{\parindent}{0pt}
\setlength{\parskip}{5pt plus 1pt}
\setlength{\headheight}{13.6pt}
\newcommand\question[2]{\vspace{.25in}\hrule\textbf{#1 #2}\vspace{.5em}\hrule\vspace{.10in}}
\renewcommand\part[1]{\vspace{.10in}\textbf{(#1)}}
\newcommand\algorithm{\vspace{.10in}\textbf{Algorithm: }}
\newcommand\correctness{\vspace{.10in}\textbf{Correctness: }}
\newcommand\runtime{\vspace{.10in}\textbf{Running time: }}
\pagestyle{fancyplain}
\lhead{\textbf{\NAME\ (\ANDREWID)}}
\chead{\textbf{Problem \HWNUM}}
\rhead{Strang., 18.06SC}
\begin{document}\raggedright
%Section A==============Change the values below to match your information==================
\newcommand\NAME{Haiying Cui}  % your name
\newcommand\ANDREWID{Christy}     % your andrew id
\newcommand\HWNUM{11}              % the homework number
%Section B==============Put your answers to the questions below here=======================

\title{Exercises on matrix spaces; rank1; small world graph}
\maketitle
% no need to restate the problem --- the graders know which problem is which,
% but replacing "The First Problem" with a short phrase will help you remember
% which problem this is when you read over your homeworks to study.

\question{11.1}{}
\(I = \begin{bmatrix} 1 0 0 \\ 0 1 0 \\ 0 0 1\end{bmatrix} = \begin{bmatrix} 0 1 0 \\ 1 0 0 \\ 0 0 1\end{bmatrix} - \begin{bmatrix} 0 1 0 \\ 0 0 1 \\ 1 0 0 \end{bmatrix} + \begin{bmatrix} 1 0 0 \\ 0 0 1 \\ 0 1 0 \end{bmatrix} - \begin{bmatrix} 0 0 1 \\ 1 0 0 \\ 0 1 0 \end{bmatrix} + \begin{bmatrix} 0 0 1 \\ 0 1 0 \\ 1 0 0\end{bmatrix}\)

If \(c_1P_1 + ... + c_5P_5 = 0 \), then \(\begin{bmatrix} c_3 \ c_1 - c_2 \ c_5-c_4 \\ c_1 - c_4 \ c_5 \ c_3 - c_2 \\ c_5 - c_2 \ c_3 - c_4 \ c_1 \end{bmatrix} = 0\). It leads to \(c_1 = c_2 = c_3 = c_4 = c_5 = 0\). Therefore, the five permutation matrices are linearly independent because the only combination of the matrices that gives the zero matrix is the zero matrix itself. 
\question{11.2}{}
\part{a}
\(A = \begin{bmatrix} 1 \ 0 \ -1 \\ -1 \ 1 \ 0 \\ 0 \ -1 \ 1 \end{bmatrix}\to\,\begin{bmatrix} 1 \ 0 \ -1 \\ 0 \ 1 \ -1 \\ 0 \ 0 \ 0 \end{bmatrix}\)

The rank of A is 2. The special solution to \(Ax = 0\) is \(\begin{bmatrix} 1 \\ 1 \\ 1 \end{bmatrix}\). Therefore, X have columns that are multiples of \(\begin{bmatrix} 1 \\ 1 \\ 1 \end{bmatrix}\). X is in the form of \(\begin{bmatrix} a \ b \ c \\ a \ b \ c \\ a \ b \ c \end{bmatrix}\).

\part{b}
For X to have the form of \(AX\), X has to be a combination of columns of \(A\). Each column of \(A\) sums up to 0, so each column of \(X\) also sums up to 0. \(X\) is in the form of \(\begin{bmatrix} a \ b \ c \\ d \ e \ f \\ -a-d \ -b-e \ -c-f \end{bmatrix}\).

\part{c}
The dimension of the the nullspace \((n - r)\)of A is 3 and the dimension of the column space \(r\) is 6. They add up to 9 which is the dimension of \(M\). 
\end{document}