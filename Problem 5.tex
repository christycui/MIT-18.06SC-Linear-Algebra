%=======================02-713 LaTeX template, following the 15-210 template==================
%
% You don't need to use LaTeX or this template, but you must turn your homework in as
% a typeset PDF somehow.
%
% How to use:
%    1. Update your information in section "A" below
%    2. Write your answers in section "B" below. Precede answers for all 
%       parts of a question with the command "\question{n}{desc}" where n is
%       the question number and "desc" is a short, one-line description of 
%       the problem. There is no need to restate the problem.
%    3. If a question has multiple parts, precede the answer to part x with the
%       command "\part{x}".
%    4. If a problem asks you to design an algorithm, use the commands
%       \algorithm, \correctness, \runtime to precede your discussion of the 
%       description of the algorithm, its correctness, and its running time, respectively.
%    5. You can include graphics by using the command \includegraphics{FILENAME}
%
\documentclass[11pt]{article}
\usepackage{amsmath,amssymb,amsthm}
\usepackage{graphicx}
\usepackage[margin=1in]{geometry}
\usepackage{fancyhdr}
\setlength{\parindent}{0pt}
\setlength{\parskip}{5pt plus 1pt}
\setlength{\headheight}{13.6pt}
\newcommand\question[2]{\vspace{.25in}\hrule\textbf{#1 #2}\vspace{.5em}\hrule\vspace{.10in}}
\renewcommand\part[1]{\vspace{.10in}\textbf{(#1)}}
\newcommand\algorithm{\vspace{.10in}\textbf{Algorithm: }}
\newcommand\correctness{\vspace{.10in}\textbf{Correctness: }}
\newcommand\runtime{\vspace{.10in}\textbf{Running time: }}
\pagestyle{fancyplain}
\lhead{\textbf{\NAME\ (\ANDREWID)}}
\chead{\textbf{Problem \HWNUM}}
\rhead{Strang., 18.06SC}
\begin{document}\raggedright
%Section A==============Change the values below to match your information==================
\newcommand\NAME{Haiying Cui}  % your name
\newcommand\ANDREWID{Christy}     % your andrew id
\newcommand\HWNUM{5}              % the homework number
%Section B==============Put your answers to the questions below here=======================

\title{Exercises on Transposes, Permutations, Spaces}
\maketitle
% no need to restate the problem --- the graders know which problem is which,
% but replacing "The First Problem" with a short phrase will help you remember
% which problem this is when you read over your homeworks to study.

\question{5.1}{}
\part{a} If we can rotate the three rows of the permutation matrix in one direction, we can get \(P^3 = I\). One example is that if \(P = \begin{bmatrix} 0 1 0 \\ 0 0 1 \\ 1 0 0 \end{bmatrix}\), then \(P^2 = \begin{bmatrix} 0 0 0 \\ 1 0 0 \\ 0 1 0 \end{bmatrix}\) and \(P^3 = \begin{bmatrix} 1 0 0 \\ 0 1 0 \\ 0 0 1 \end{bmatrix}\).

\part{b} If we can exchange the rows in a 4 by 4 matrix in an non-orderly fashion, unlike \(P\) in part (a), then we will get \(P^4 \neq I\). For example, if \(\hat{P} = \begin{bmatrix} 0 1 0 0 \\ 0 0 0 1 \\ 0 0 1 0 \\ 1 0 0 0 \end{bmatrix}\), then \(\hat{P}^2 = \begin{bmatrix} 0 0 0 1 \\ 1 0 0 0 \\ 0 0 1 0 \\ 0 1 0 0 \end{bmatrix}\), \(\hat{P}^3 = \begin{bmatrix} 1 0 0 0 \\ 0 1 0 0 \\ 0 0 1 0 \\ 0 0 0 1 \end{bmatrix}\) and \(\hat{P}^4 = \begin{bmatrix} 0 1 0 0 \\ 0 0 0 1 \\ 0 0 1 0 \\ 1 0 0 0 \end{bmatrix} \neq I\).

\question{5.2}{}
\part{a} If \(A\) is symmetric, then we can represent \(A\) by:
$$\begin{bmatrix} a \ b \ c \ d \\ b \ e \ h \ i \\ c \ h \ f \ j \\ d \ i \ j \ g \end{bmatrix}$$
There are 10 independent entries.

\part{b} If \(A\) is skew-symmetric (\(A^{T} = - A\)), we can represent \(A\) by:
$$\begin{bmatrix} 0 \ a \ b \ c \\ -a \ 0 \ d \ e \\ -b \ -d \ 0 \ f \\ -c \ -e \ -f \ 0 \end{bmatrix}$$
There are 6 independent entries.

\question{5.3}{}
\part{a} \textbf{True}: With \(A^T = A\) and \(B^T = B\), it leads us to \((A + B)^{T} = A^T + B^T = A + B\) and \((cA)^T = cA^T\). The addition and scalar multiplication results are both in the subspace. 

\part{b} \textbf{True}: With \(A^T = -A\), it leads to \((A + B)^T = A^T + B^T= -A - B = - (A + B) \) and \((cA)^T = cA^T = -cA\).

\part{c} \textbf{Flase}: With \(A^T \neq A\) and \(B^T \neq B\), it leads us to \((A + B)^{T} = A^T + B^T \neq A + B\). An example is:
$$\begin{bmatrix} 1 0 \\ 1 0 \end{bmatrix} + \begin{bmatrix} 0 1 \\ 0 1 \end{bmatrix} = \begin{bmatrix} 1 1 \\ 1 1 \end{bmatrix}$$
It does not form a subspace, because the result of addition is a symmetrical matrix.

\end{document}
