%=======================02-713 LaTeX template, following the 15-210 template==================
%
% You don't need to use LaTeX or this template, but you must turn your homework in as
% a typeset PDF somehow.
%
% How to use:
%    1. Update your information in section "A" below
%    2. Write your answers in section "B" below. Precede answers for all 
%       parts of a question with the command "\question{n}{desc}" where n is
%       the question number and "desc" is a short, one-line description of 
%       the problem. There is no need to restate the problem.
%    3. If a question has multiple parts, precede the answer to part x with the
%       command "\part{x}".
%    4. If a problem asks you to design an algorithm, use the commands
%       \algorithm, \correctness, \runtime to precede your discussion of the 
%       description of the algorithm, its correctness, and its running time, respectively.
%    5. You can include graphics by using the command \includegraphics{FILENAME}
%
\documentclass[11pt]{article}
\usepackage{amsmath,amssymb,amsthm}
\usepackage{graphicx}
\usepackage[margin=1in]{geometry}
\usepackage{fancyhdr}
\setlength{\parindent}{0pt}
\setlength{\parskip}{5pt plus 1pt}
\setlength{\headheight}{13.6pt}
\newcommand\question[2]{\vspace{.25in}\hrule\textbf{#1 #2}\vspace{.5em}\hrule\vspace{.10in}}
\renewcommand\part[1]{\vspace{.10in}\textbf{(#1)}}
\newcommand\algorithm{\vspace{.10in}\textbf{Algorithm: }}
\newcommand\correctness{\vspace{.10in}\textbf{Correctness: }}
\newcommand\runtime{\vspace{.10in}\textbf{Running time: }}
\pagestyle{fancyplain}
\lhead{\textbf{\NAME\ (\ANDREWID)}}
\chead{\textbf{Problem \HWNUM}}
\rhead{Strang., 18.06SC}
\begin{document}\raggedright
%Section A==============Change the values below to match your information==================
\newcommand\NAME{Haiying Cui}  % your name
\newcommand\ANDREWID{Christy}     % your andrew id
\newcommand\HWNUM{24}              % the homework number
%Section B==============Put your answers to the questions below here=======================

\title{Exercises on Markov matrices; Fourier series}
\maketitle
% no need to restate the problem --- the graders know which problem is which,
% but replacing "The First Problem" with a short phrase will help you remember
% which problem this is when you read over your homeworks to study.

\question{24.1}{}
\part{a} $$\begin{bmatrix} 1 \ b \\ b \ 1 \end{bmatrix}\ \,\to\, \begin{bmatrix} 1 \ b \\ 0 \ 1-b^2\end{bmatrix}$$

\((1+b)(1-b) < 0 \,\to\, b < -1\) or \(b > 1\). 

One symmetric matrix that has a negative eiganvalue is \(\begin{bmatrix} 1 \ -2 \\ 0 \ -3 \end{bmatrix}\).

\part{b} Since one of the pivots is 1 after elimination, the other pivot (eiganvalue) must be negative.

\part{c} It can't have two negative eigan values because there are only two cases, \(b < -1\) or \(b > 1\). No matter which case is \(b\) in, one of the terms in \((1+b)(1-b)\) is negative and one is positive.

\question{24.2}{}
\(A\) belongs to the following classes invertible, orthogonal, permutation, diagonalizable and Markov. \(A\) is not a projection because \(A\) times itself does not result \(A\). 

\(B\) belongs to diagonalizable and Markov. \(B\) is not invertible because there are dependent columns. \(B\) is not orthogonal and is not a permutation matrix. \(B\) is a projection because \(B^2 = B\).

\(QR,S\Lambda S^{-1}, Q\Lambda Q^T\) is possible for \(A\); and \(LU, S\Lambda S^{-1}, Q\Lambda Q^T\) is possible for \(B\).

\question{24.3}{}
Since \(A\) is a Markov matrix, every column of \(A\) sums up to 1.
$$A = \begin{bmatrix} .7 \ .1 \ .2 \\ .1 \ .6 \ .3 \\ .2 \ .3 \ .5 \end{bmatrix}$$
We know that one of \(A\)'s \(\lambda\) is 1. And the eiganvector when \(\lambda\) is one is the steady state. \((A-\lambda I)x = \begin{bmatrix} -.3 \ .1 \ .2 \\ .1 \ -.4 \ .3 \\ .2 \ .3 \ -.5 \end{bmatrix}x = 0 \,\to\, x = \begin{bmatrix} 1 \\ 1 \\ 1 \end{bmatrix}\)

When \(A\) is a symmetric Markov matrix, not only every column sums up to one, every row also sums up to one. So when computing the eiganvector from \((A-\lambda I)\), each row of that matrix sums up to zero. Therefore \(x_1 = (1, ... , 1)\).
\end{document}