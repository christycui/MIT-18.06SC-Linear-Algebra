%=======================02-713 LaTeX template, following the 15-210 template==================
%
% You don't need to use LaTeX or this template, but you must turn your homework in as
% a typeset PDF somehow.
%
% How to use:
%    1. Update your information in section "A" below
%    2. Write your answers in section "B" below. Precede answers for all 
%       parts of a question with the command "\question{n}{desc}" where n is
%       the question number and "desc" is a short, one-line description of 
%       the problem. There is no need to restate the problem.
%    3. If a question has multiple parts, precede the answer to part x with the
%       command "\part{x}".
%    4. If a problem asks you to design an algorithm, use the commands
%       \algorithm, \correctness, \runtime to precede your discussion of the 
%       description of the algorithm, its correctness, and its running time, respectively.
%    5. You can include graphics by using the command \includegraphics{FILENAME}
%
\documentclass[11pt]{article}
\usepackage{amsmath,amssymb,amsthm}
\usepackage{graphicx}
\usepackage[margin=1in]{geometry}
\usepackage{fancyhdr}
\setlength{\parindent}{0pt}
\setlength{\parskip}{5pt plus 1pt}
\setlength{\headheight}{13.6pt}
\newcommand\question[2]{\vspace{.25in}\hrule\textbf{#1 #2}\vspace{.5em}\hrule\vspace{.10in}}
\renewcommand\part[1]{\vspace{.10in}\textbf{(#1)}}
\newcommand\algorithm{\vspace{.10in}\textbf{Algorithm: }}
\newcommand\correctness{\vspace{.10in}\textbf{Correctness: }}
\newcommand\runtime{\vspace{.10in}\textbf{Running time: }}
\pagestyle{fancyplain}
\lhead{\textbf{\NAME\ (\ANDREWID)}}
\chead{\textbf{Problem \HWNUM}}
\rhead{Strang., 18.06SC}
\begin{document}\raggedright
%Section A==============Change the values below to match your information==================
\newcommand\NAME{Haiying Cui}  % your name
\newcommand\ANDREWID{Christy}     % your andrew id
\newcommand\HWNUM{23}              % the homework number
%Section B==============Put your answers to the questions below here=======================

\title{Exercises on differential equations and \(e^{At}\)}
\maketitle
% no need to restate the problem --- the graders know which problem is which,
% but replacing "The First Problem" with a short phrase will help you remember
% which problem this is when you read over your homeworks to study.

\question{23.1}{}
\(\frac{d||u(t)||^2}{dt} = \frac{d(u_1^2 + u_2^2 + u_3^2)}{dt} = 2u_1u_1' + 2u_2u_2' + 2u_3u_3' = 2u_1(cu_2-bu_3) + 2u_2(au_3 - cu_1) + 2u_3(bu_1-au_2) = 0\)

This means that the length of \(u(t)\) does not change. 

\question{23.2}{}
Diagonalize \(A\):
$$det (A-\lambda I) = det \begin{bmatrix} 1-\lambda 1 \\ 0 \ 3-\lambda \end{bmatrix} = (1-\lambda)(3-\lambda) = 0$$
$$\lambda = 1, 3$$
When \(\lambda = 1\), \((A-\lambda I)x = \begin{bmatrix} 0 \ 1 \\ 0 \ 2 \end{bmatrix} x = 0 \,\to\, x = \begin{bmatrix} 1 \\ 0 \end{bmatrix}\).

When \(\lambda = 3\), \((A-\lambda I)x = \begin{bmatrix} -2 \ 1 \\ 0 \ 0 \end{bmatrix} x = 0 \,\to\, x = \begin{bmatrix} 1 \\ 2 \end{bmatrix}\).

Therefore, \(A = \begin{bmatrix} 1 \ 1 \\ 0 \ 3 \end{bmatrix} = S\Lambda S^{-1}\) where \(S = \begin{bmatrix} 1 \ 1 \\ 0 \ 2 \end{bmatrix}\) and \(\Lambda = \begin{bmatrix} 1 \ 0 \\ 0 \ 3 \end{bmatrix}\).

$$e^{At} = Se^{\Lambda t}S^{-1} = \begin{bmatrix} 1 \ 1 \\ 0 \ 2 \end{bmatrix}\begin{bmatrix} e^t \ 0 \\ 0 \ e^{3t} \end{bmatrix}\cdot \frac{1}{2} \cdot \begin{bmatrix} 2 \ -1 \\ 0 \ 1 \end{bmatrix} = \begin{bmatrix} e^t \ e^{3t} \\ 0 \ 2e^{3t} \end{bmatrix} \cdot \frac{1}{2} \cdot \begin{bmatrix} 2 \ -1 \\ 0 \ 1 \end{bmatrix} = \frac{1}{2} \begin{bmatrix} 2e^t \ -e^t+e^{3t} \\ 0 \ 2e^{3t} \end{bmatrix}$$
$$e^{At}|_{t=0} = \frac{1}{2} \begin{bmatrix} 2 \ 0 \\ 0 \ 2 \end{bmatrix} = \begin{bmatrix} 1 \ 0 \\ 0 \ 1 \end{bmatrix} = I $$
$$\frac{d(d^{At})}{dt} = \frac{1}{2} \begin{bmatrix} 2e^t \ -e^t+3e^{3t} \\ 0 \ 6e^{3t} \end{bmatrix} = \frac{1}{2} \begin{bmatrix} 2 \ 2 \\ 0 \ 6 \end{bmatrix} = \begin{bmatrix} 1 \ 1 \\ 0 \ 3 \end{bmatrix} = A $$
$$\frac{ud}{dt} = Au \checkmark $$
\end{document}