%=======================02-713 LaTeX template, following the 15-210 template==================
%
% You don't need to use LaTeX or this template, but you must turn your homework in as
% a typeset PDF somehow.
%
% How to use:
%    1. Update your information in section "A" below
%    2. Write your answers in section "B" below. Precede answers for all 
%       parts of a question with the command "\question{n}{desc}" where n is
%       the question number and "desc" is a short, one-line description of 
%       the problem. There is no need to restate the problem.
%    3. If a question has multiple parts, precede the answer to part x with the
%       command "\part{x}".
%    4. If a problem asks you to design an algorithm, use the commands
%       \algorithm, \correctness, \runtime to precede your discussion of the 
%       description of the algorithm, its correctness, and its running time, respectively.
%    5. You can include graphics by using the command \includegraphics{FILENAME}
%
\documentclass[11pt]{article}
\usepackage{amsmath,amssymb,amsthm}
\usepackage{graphicx}
\usepackage[margin=1in]{geometry}
\usepackage{fancyhdr}
\setlength{\parindent}{0pt}
\setlength{\parskip}{5pt plus 1pt}
\setlength{\headheight}{13.6pt}
\newcommand\question[2]{\vspace{.25in}\hrule\textbf{#1 #2}\vspace{.5em}\hrule\vspace{.10in}}
\renewcommand\part[1]{\vspace{.10in}\textbf{(#1)}}
\newcommand\algorithm{\vspace{.10in}\textbf{Algorithm: }}
\newcommand\correctness{\vspace{.10in}\textbf{Correctness: }}
\newcommand\runtime{\vspace{.10in}\textbf{Running time: }}
\pagestyle{fancyplain}
\lhead{\textbf{\NAME\ (\ANDREWID)}}
\chead{\textbf{Problem \HWNUM}}
\rhead{Strang., 18.06SC}
\begin{document}\raggedright
%Section A==============Change the values below to match your information==================
\newcommand\NAME{Haiying Cui}  % your name
\newcommand\ANDREWID{Christy}     % your andrew id
\newcommand\HWNUM{12}              % the homework number
%Section B==============Put your answers to the questions below here=======================

\title{Exercises on graphs, networks, and incidence matrices}
\maketitle
% no need to restate the problem --- the graders know which problem is which,
% but replacing "The First Problem" with a short phrase will help you remember
% which problem this is when you read over your homeworks to study.

\question{12.1}{}
Incidence matrix \(A = \begin{bmatrix} -1 \ 1 \ 0 \ 0 \\ 0 \ -1 \ 1 \ 0 \\ 0 \ 0 \ 1 \ -1 \\ -1 \ 0 \ 0 \ 1 \end{bmatrix}\). Vector \(\begin{bmatrix} 1 \\ 1 \\ 1 \\ 1 \end{bmatrix}\) is in the nullspace of \(A\). (1,0,0,0) is not in the row space of \(A\) because a row of incidence matrix represents the nodes through which the current flows and there are two nonzero elements instead of one in a row.
\question{12.2}{}
To find \(A^TCA\):

\(A^TCA = \begin{bmatrix} -1 \ 0 \ 0 \ -1 \\ 1 \ -1 \ 0 \ 0 \\ 0 \ 1 \ 1 \ 0 \\ 0 \ 0 \ -1 \ 1 \end{bmatrix}\begin{bmatrix} 1 \ 0 \ 0 \ 0 \\ 0 \ 2 \ 0 \ 0 \\ 0 \ 0 \ 2 \ 0 \\ 0 \ 0 \ 0 \ 1 \end{bmatrix}\begin{bmatrix} -1 \ 1 \ 0 \ 0 \\ 0 \ -1 \ 1 \ 0 \\ 0 \ 0 \ 1 \ -1 \\ -1 \ 0 \ 0 \ 1 \end{bmatrix} = \begin{bmatrix} -1 \ 0 \ 0 \ -1 \\ 1 \ -2 \ 0 \ 0 \\ 0 \ 2 \ 2 \ 0 \\ 0 \ 0 \ -2 \ 1 \end{bmatrix}\begin{bmatrix} -1 \ 1 \ 0 \ 0 \\ 0 \ -1 \ 1 \ 0 \\ 0 \ 0 \ 1 \ -1 \\ -1 \ 0 \ 0 \ 1 \end{bmatrix} = \begin{bmatrix} 2 \ -1 \ 0 \ -1 \\ -1 \ 3 \ -2 \ 0 \\ 0 \ -2 \ 4 \ -2 \\ -1 \ 0 \ -2 \ 3 \end{bmatrix}\)

Since \(A^TCAx = f\), we can find \(x\) by doing elimination:

\(\begin{bmatrix}\begin{array}{cccc|c} -1 & 1 & 0 & 0 & 1 \\ 0 & -1 & 1 & 0 & 0 \\ 0 & 0 & 1 & -1 & -1 \\ -1 & 0 & 0 & 1 & 0 \end{array}\end{bmatrix}\,\to\,\begin{bmatrix}\begin{array}{cccc|c} 1 & -\frac{1}{2} & 0 & -\frac{1}{2} & \frac{1}{2} \\ 0 & \frac{5}{2} & -2 & -\frac{1}{2} & \frac{1}{2} \\ 0 & -2 & 4 & -2 & -1 \\ 0 & -\frac{1}{2} & -2 & \frac{5}{2} & \frac{1}{2} \end{array}\end{bmatrix}\,\to\,\begin{bmatrix}\begin{array}{cccc|c} 1 & -\frac{1}{2} & 0 & -\frac{1}{2} & \frac{1}{2} \\ 0 & 1 & -\frac{4}{5} & -\frac{1}{5} & \frac{1}{5} \\ 0 & 0 & \frac{12}{5} & -\frac{12}{5} & -\frac{3}{5} \\ 0 & 0 & -\frac{12}{5} & \frac{12}{5} & \frac{3}{5} \end{array}\end{bmatrix}\,\to\,\begin{bmatrix}\begin{array}{cccc|c} 1 & -\frac{1}{2} & 0 & -\frac{1}{2} & \frac{1}{2} \\ 0 & 1 & -\frac{4}{5} & -\frac{1}{5} & \frac{1}{5} \\ 0 & 0 & 1 & -1 & -\frac{1}{4} \\ 0 & 0 & 0 & 0 & 0 \end{array}\end{bmatrix}\)

One solutions is \(x = \begin{bmatrix} \frac{3}{2} \\ 1 \\ \frac{3}{4} \\ 1 \end{bmatrix}\).

Since current \(y = -CAx\), \(y = \begin{bmatrix} -1 \ 0 \ 0 \ 0 \\ 0 \ -2 \ 0 \ 0 \\ 0 \ 0 \ -2 \ 0 \\ 0 \ 0 \ 0 \ -1 \end{bmatrix}\begin{bmatrix} -1 \ 1 \ 0 \ 0 \\ 0 \ -1 \ 1 \ 0 \\ 0 \ 0 \ 1 \ -1 \\ -1 \ 0 \ 0 \ 1 \end{bmatrix}\begin{bmatrix}\frac{3}{2} \\ 1 \\ \frac{3}{4} \\ 1 \end{bmatrix} = \begin{bmatrix} 1 \ -1 \ 0 \ 0 \\ 0 \ 2 \ -2 \ 0 \\ 0 \ 0 \ -2 \ 2 \\ 1 \ 0 \ 0 \ -1 \end{bmatrix}\begin{bmatrix}\frac{3}{2} \\ 1 \\ \frac{3}{4} \\ 1 \end{bmatrix} = \begin{bmatrix}\frac{1}{2} \\ \frac{1}{2} \\ \frac{1}{2} \\ \frac{1}{2} \end{bmatrix}\)

\end{document}