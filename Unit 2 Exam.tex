%=======================02-713 LaTeX template, following the 15-210 template==================
%
% You don't need to use LaTeX or this template, but you must turn your homework in as
% a typeset PDF somehow.
%
% How to use:
%    1. Update your information in section "A" below
%    2. Write your answers in section "B" below. Precede answers for all 
%       parts of a question with the command "\question{n}{desc}" where n is
%       the question number and "desc" is a short, one-line description of 
%       the problem. There is no need to restate the problem.
%    3. If a question has multiple parts, precede the answer to part x with the
%       command "\part{x}".
%    4. If a problem asks you to design an algorithm, use the commands
%       \algorithm, \correctness, \runtime to precede your discussion of the 
%       description of the algorithm, its correctness, and its running time, respectively.
%    5. You can include graphics by using the command \includegraphics{FILENAME}
%
\documentclass[11pt]{article}
\usepackage{amsmath,amssymb,amsthm}
\usepackage{graphicx}
\usepackage[margin=1in]{geometry}
\usepackage{fancyhdr}
\setlength{\parindent}{0pt}
\setlength{\parskip}{5pt plus 1pt}
\setlength{\headheight}{13.6pt}
\newcommand\question[2]{\vspace{.25in}\hrule\textbf{#1 #2}\vspace{.5em}\hrule\vspace{.10in}}
\renewcommand\part[1]{\vspace{.10in}\textbf{(#1)}}
\newcommand\algorithm{\vspace{.10in}\textbf{Algorithm: }}
\newcommand\correctness{\vspace{.10in}\textbf{Correctness: }}
\newcommand\runtime{\vspace{.10in}\textbf{Running time: }}
\pagestyle{fancyplain}
\lhead{\textbf{\NAME\ (\ANDREWID)}}
\chead{\textbf{ \HWNUM}}
\rhead{Strang., 18.06SC}
\begin{document}\raggedright
%Section A==============Change the values below to match your information==================
\newcommand\NAME{Haiying Cui}  % your name
\newcommand\ANDREWID{Christy}     % your andrew id
\newcommand\HWNUM{}              % the homework number
%Section B==============Put your answers to the questions below here=======================

\title{Unit 2 Exam}
\maketitle
% no need to restate the problem --- the graders know which problem is which,
% but replacing "The First Problem" with a short phrase will help you remember
% which problem this is when you read over your homeworks to study.

\question{1}{}
\part{a} \(\pm1\)

Orthogonal matrix has determinant of \(\pm1\).

\part{b} The determinant is a linear function of each row; therefore,
$$det \begin{bmatrix} q_1+q_2 \ q_2+q_3 \ a_3+q_1 \end{bmatrix} = \begin{bmatrix} q_1 \ q_2 \ q_3 \end{bmatrix} + \begin{bmatrix} q_2 \ q_3 \ q_1 \end{bmatrix} = \pm2$$

\part{c} The second term is the same as the first term because it requires two row exchanges (changes sign twice), so the product is always 1.

\question{2}{}
\part{a} We can rewrite the problem as
$$\begin{bmatrix} 1 \ -10 \\ 1 \ -9 \\ \vdots \\ 1 \ 0 \\ \vdots \\ 1 \ 10 \end{bmatrix}\begin{bmatrix} C \\ D \end{bmatrix} = \begin{bmatrix} 0 \\ 0 \\ \vdots \\ 1 \\ \vdots \\ 0 \end{bmatrix}$$
Multiply both sides by \(A^T\)
$$\begin{bmatrix} 21 \ 0 \\ 0 \ 770 \end{bmatrix}\begin{bmatrix} C \\ D \end{bmatrix} = \begin{bmatrix} 1 \\ 0 \end{bmatrix}$$
$$\begin{bmatrix} C \\ D \end{bmatrix} = \begin{bmatrix} \frac{1}{21} \\ 0 \end{bmatrix}$$

\part{b} I am projecting the matrix onto the column space of above. 

The two bases are:
\(\begin{bmatrix} 1 \\ \vdots \\  \\ 1 \end{bmatrix}\) and \(\begin{bmatrix} -10 \\ -9 \\ \vdots \\ 0 \\ \vdots \\ 10 \end{bmatrix}\)

The error vector is perpendicular to the subspace. 
$$e = b - Pb = \begin{bmatrix} 0 \\ \vdots \\ 1 \\ \vdots \\ 0 \end{bmatrix} - \begin{bmatrix} \frac{1}{21} \\ \vdots \\ \frac{1}{21} \end{bmatrix} = \begin{bmatrix} - \frac{1}{21} \\ \vdots \\ \frac{20}{21} \\ \vdots \\ -\frac{1}{21} \end{bmatrix}$$
\question{3}{}
\part{a} 
$$P_A = A(A^TA)^{-1}A^T$$
$$P_Q = Q(Q^TQ)^{-1}Q^T$$

\part{b} Because both projection matrices project onto the same subspace. \(P_QQ\) is the projection of \(Q\) on to the column space of \(Q\). Determinant of \(P_Q\) is zero because there are only 3 independent vectors in \(R^5\) aka \(P_Q\) is singular. 

\part{c} Choice 3

\question{4}{}
\part{a} Since the determinant can only come from one entry per column per row, the largest possible degree of polynomial is two.

\part{b} det \(A = x(1) -x(x) + x(-x) - x (-1) (-x) = x - 3x^2\) 

When x = 3 or x = \(\frac{1}{3}\), det \(A\) = 0.
\end{document}