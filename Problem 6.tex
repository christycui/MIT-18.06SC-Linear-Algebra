%=======================02-713 LaTeX template, following the 15-210 template==================
%
% You don't need to use LaTeX or this template, but you must turn your homework in as
% a typeset PDF somehow.
%
% How to use:
%    1. Update your information in section "A" below
%    2. Write your answers in section "B" below. Precede answers for all 
%       parts of a question with the command "\question{n}{desc}" where n is
%       the question number and "desc" is a short, one-line description of 
%       the problem. There is no need to restate the problem.
%    3. If a question has multiple parts, precede the answer to part x with the
%       command "\part{x}".
%    4. If a problem asks you to design an algorithm, use the commands
%       \algorithm, \correctness, \runtime to precede your discussion of the 
%       description of the algorithm, its correctness, and its running time, respectively.
%    5. You can include graphics by using the command \includegraphics{FILENAME}
%
\documentclass[11pt]{article}
\usepackage{amsmath,amssymb,amsthm}
\usepackage{graphicx}
\usepackage[margin=1in]{geometry}
\usepackage{fancyhdr}
\setlength{\parindent}{0pt}
\setlength{\parskip}{5pt plus 1pt}
\setlength{\headheight}{13.6pt}
\newcommand\question[2]{\vspace{.25in}\hrule\textbf{#1 #2}\vspace{.5em}\hrule\vspace{.10in}}
\renewcommand\part[1]{\vspace{.10in}\textbf{(#1)}}
\newcommand\algorithm{\vspace{.10in}\textbf{Algorithm: }}
\newcommand\correctness{\vspace{.10in}\textbf{Correctness: }}
\newcommand\runtime{\vspace{.10in}\textbf{Running time: }}
\pagestyle{fancyplain}
\lhead{\textbf{\NAME\ (\ANDREWID)}}
\chead{\textbf{Problem \HWNUM}}
\rhead{Strang., 18.06SC}
\begin{document}\raggedright
%Section A==============Change the values below to match your information==================
\newcommand\NAME{Haiying Cui}  % your name
\newcommand\ANDREWID{Christy}     % your andrew id
\newcommand\HWNUM{6}              % the homework number
%Section B==============Put your answers to the questions below here=======================

\title{Exercises on Column Space and Nullspace}
\maketitle
% no need to restate the problem --- the graders know which problem is which,
% but replacing "The First Problem" with a short phrase will help you remember
% which problem this is when you read over your homeworks to study.

\question{6.1}{}
\part{a} Suppose there is another vector in the space \(S + T\): \(s' + t'\) where \(s'\) is in \(S\) and \(t'\) is in \(T\). 
\((s + t) + (s' + t') = (s + s') + (t + t')\) and \(c(s + t) = cs + ct\). \(S + T\) is a vector space because it satisfies the two requirements.

\part{b} If \(S\) and \(T\) are lines, then \(S + T\) is a plane and \(S \cup T\) is two lines. The plane contains all vectors in the two lines. \(S + T\) contains all combinations of the vectors in \(S\) and \(T\). Therefore, \(S + T\) is a span of \(S \cup T\).

\question{6.2}{}
We can rewrite the first equation to \(x = 12 +3y + z\). So the solution is 12, 3, 1.

\question{6.3}{} \(N(C)\) is all the solutions to the equation
$$Nx = \begin{bmatrix}Ax \\ Bx \end{bmatrix} = 0$$

\(N(C)\) has to satisfy both \(Ax = 0\) and \(Bx = 0\). Therefore, \(N(C) = N(A) \cap N(B)\).
\end{document}
