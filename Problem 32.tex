%=======================02-713 LaTeX template, following the 15-210 template==================
%
% You don't need to use LaTeX or this template, but you must turn your homework in as
% a typeset PDF somehow.
%
% How to use:
%    1. Update your information in section "A" below
%    2. Write your answers in section "B" below. Precede answers for all 
%       parts of a question with the command "\question{n}{desc}" where n is
%       the question number and "desc" is a short, one-line description of 
%       the problem. There is no need to restate the problem.
%    3. If a question has multiple parts, precede the answer to part x with the
%       command "\part{x}".
%    4. If a problem asks you to design an algorithm, use the commands
%       \algorithm, \correctness, \runtime to precede your discussion of the 
%       description of the algorithm, its correctness, and its running time, respectively.
%    5. You can include graphics by using the command \includegraphics{FILENAME}
%
\documentclass[11pt]{article}
\usepackage{amsmath,amssymb,amsthm}
\usepackage{graphicx}
\usepackage[margin=1in]{geometry}
\usepackage{fancyhdr}
\setlength{\parindent}{0pt}
\setlength{\parskip}{5pt plus 1pt}
\setlength{\headheight}{13.6pt}
\newcommand\question[2]{\vspace{.25in}\hrule\textbf{#1 #2}\vspace{.5em}\hrule\vspace{.10in}}
\renewcommand\part[1]{\vspace{.10in}\textbf{(#1)}}
\newcommand\algorithm{\vspace{.10in}\textbf{Algorithm: }}
\newcommand\correctness{\vspace{.10in}\textbf{Correctness: }}
\newcommand\runtime{\vspace{.10in}\textbf{Running time: }}
\pagestyle{fancyplain}
\lhead{\textbf{\NAME\ (\ANDREWID)}}
\chead{\textbf{Problem \HWNUM}}
\rhead{Strang., 18.06SC}
\begin{document}\raggedright
%Section A==============Change the values below to match your information==================
\newcommand\NAME{Haiying Cui}  % your name
\newcommand\ANDREWID{Christy}     % your andrew id
\newcommand\HWNUM{32}              % the homework number
%Section B==============Put your answers to the questions below here=======================

\title{Exercises on left and right inverses; pseudoinverse}
\maketitle
% no need to restate the problem --- the graders know which problem is which,
% but replacing "The First Problem" with a short phrase will help you remember
% which problem this is when you read over your homeworks to study.

\question{32.1}{}
$$A^{-1}_{right} = A^T(AA^T)^{-1}$$
$$AA^T = \begin{bmatrix} 1 \ 0 \ 1 \\ 0 \ 1 \ 0 \end{bmatrix}\begin{bmatrix} 1 \ 0 \\ 0 \ 1 \\ 1 \ 0 \end{bmatrix} = \begin{bmatrix} 2 \ 0 \\ 0 \ 1 \end{bmatrix}$$
$$(AA^T)^{-1} = \frac{1}{2}\begin{bmatrix} 1 \ 0 \\ 0 \ 2 \end{bmatrix}$$
$$A^{-1}_{right} = \frac{1}{2}\begin{bmatrix} 1 \ 0 \\ 0 \ 1 \\ 1 \ 0 \end{bmatrix}\begin{bmatrix} 1 \ 0 \\ 0 \ 2 \end{bmatrix} = \frac{1}{2}\begin{bmatrix} 1 \ 0 \\ 0 \ 2 \\ 1 \ 0 \end{bmatrix}$$

\question{32.2}{}
By looking at \(A\), we can see that \(A\) is not invertible because the second column is a multiple of the first column. To find the pserdoinverse of \(A\), first we have to find the decomposition of \(A\) it self.
$$A^TA = \begin{bmatrix} 4 \ 8 \\ 3 \ 6 \end{bmatrix}\begin{bmatrix} 4 \ 3 \\ 8 \ 6 \end{bmatrix} = \begin{bmatrix} 80 \ 60 \\ 60 \ 45 \end{bmatrix}$$
$$det(\begin{bmatrix} 80-\lambda \ 60 \\ 60 \ 45-\lambda \end{bmatrix}) = (80-\lambda)(45-\lambda)-360 = \lambda^2 - 125\lambda = 0$$
$$\lambda = 0, 125$$
When \(\lambda = 125\), the eiganvector of \(A^TA\) is \(\begin{bmatrix} 0.8 \\ 0.6 \end{bmatrix}\). When \(\lambda = 0\), the eiganvector is \(\begin{bmatrix} 0.6 \\ -0.8 \end{bmatrix}\).

\(u_1 = Av_1 = \begin{bmatrix} 1 \\ 2 \end{bmatrix}\) and \(u_2 = Av_2 = \begin{bmatrix} 2 \\ -1 \end{bmatrix}\)

$$A^{+} = V\Sigma^{+}U^T = \begin{bmatrix} 0.8 \ 0.6 \\ 0.6 \ -0.8 \end{bmatrix}\begin{bmatrix} \sqrt{125} \ 0 \\ 0 \ 0 \end{bmatrix}\frac{1}{\sqrt{5}}\begin{bmatrix} 1 \ 2 \\ 2 \ -1 \end{bmatrix} = \frac{1}{125}\begin{bmatrix} 4 \ 8 \\ 3 \ 6 \end{bmatrix}$$
\end{document}