%=======================02-713 LaTeX template, following the 15-210 template==================
%
% You don't need to use LaTeX or this template, but you must turn your homework in as
% a typeset PDF somehow.
%
% How to use:
%    1. Update your information in section "A" below
%    2. Write your answers in section "B" below. Precede answers for all 
%       parts of a question with the command "\question{n}{desc}" where n is
%       the question number and "desc" is a short, one-line description of 
%       the problem. There is no need to restate the problem.
%    3. If a question has multiple parts, precede the answer to part x with the
%       command "\part{x}".
%    4. If a problem asks you to design an algorithm, use the commands
%       \algorithm, \correctness, \runtime to precede your discussion of the 
%       description of the algorithm, its correctness, and its running time, respectively.
%    5. You can include graphics by using the command \includegraphics{FILENAME}
%
\documentclass[11pt]{article}
\usepackage{amsmath,amssymb,amsthm}
\usepackage{graphicx}
\usepackage[margin=1in]{geometry}
\usepackage{fancyhdr}
\setlength{\parindent}{0pt}
\setlength{\parskip}{5pt plus 1pt}
\setlength{\headheight}{13.6pt}
\newcommand\question[2]{\vspace{.25in}\hrule\textbf{#1 #2}\vspace{.5em}\hrule\vspace{.10in}}
\renewcommand\part[1]{\vspace{.10in}\textbf{(#1)}}
\newcommand\algorithm{\vspace{.10in}\textbf{Algorithm: }}
\newcommand\correctness{\vspace{.10in}\textbf{Correctness: }}
\newcommand\runtime{\vspace{.10in}\textbf{Running time: }}
\pagestyle{fancyplain}
\lhead{\textbf{\NAME\ (\ANDREWID)}}
\chead{\textbf{Problem \HWNUM}}
\rhead{Strang., 18.06SC}
\begin{document}\raggedright
%Section A==============Change the values below to match your information==================
\newcommand\NAME{Haiying Cui}  % your name
\newcommand\ANDREWID{Christy}     % your andrew id
\newcommand\HWNUM{27}              % the homework number
%Section B==============Put your answers to the questions below here=======================

\title{Exercises on positive definite matrices and minima}
\maketitle
% no need to restate the problem --- the graders know which problem is which,
% but replacing "The First Problem" with a short phrase will help you remember
% which problem this is when you read over your homeworks to study.

\question{27.1}{}
Starting with the equation \(Ax = \lambda x\), we want to prove that \(\lambda > 0\).
$$(ABx)^TBx = (\lambda x)^TBx$$
$$(Bx)^TA(Bx) = \lambda (x^TBx)$$
Since both \(A\) and \(B\) are positive definite, we know that \(x^TAx > 0\) and \(x^TBx > 0\). Given the left hand side is positive and one of the factors of right hand side is positive, it leads us to \(\lambda > 0\).

\question{27.2}{}
$$x^TAx = \begin{bmatrix} x \ y \end{bmatrix}\begin{bmatrix} 1 \ 5 \\ 7 \ 9 \end{bmatrix}\begin{bmatrix} x \\ y \end{bmatrix} =  \begin{bmatrix} x+7y \ 5x+9y \end{bmatrix} \begin{bmatrix} x \\ y \end{bmatrix} = x^2 + 12xy + 9y^2 = (x+3y)^2 +6xy$$

The energy of \(A\) is sometimes positive sometimes negative, because the first term is always going to be equal or greater than zero while the second term is not.
\end{document}