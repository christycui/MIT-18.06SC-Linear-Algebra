%=======================02-713 LaTeX template, following the 15-210 template==================
%
% You don't need to use LaTeX or this template, but you must turn your homework in as
% a typeset PDF somehow.
%
% How to use:
%    1. Update your information in section "A" below
%    2. Write your answers in section "B" below. Precede answers for all 
%       parts of a question with the command "\question{n}{desc}" where n is
%       the question number and "desc" is a short, one-line description of 
%       the problem. There is no need to restate the problem.
%    3. If a question has multiple parts, precede the answer to part x with the
%       command "\part{x}".
%    4. If a problem asks you to design an algorithm, use the commands
%       \algorithm, \correctness, \runtime to precede your discussion of the 
%       description of the algorithm, its correctness, and its running time, respectively.
%    5. You can include graphics by using the command \includegraphics{FILENAME}
%
\documentclass[11pt]{article}
\usepackage{amsmath,amssymb,amsthm}
\usepackage{graphicx}
\usepackage[margin=1in]{geometry}
\usepackage{fancyhdr}
\setlength{\parindent}{0pt}
\setlength{\parskip}{5pt plus 1pt}
\setlength{\headheight}{13.6pt}
\newcommand\question[2]{\vspace{.25in}\hrule\textbf{#1 #2}\vspace{.5em}\hrule\vspace{.10in}}
\renewcommand\part[1]{\vspace{.10in}\textbf{(#1)}}
\newcommand\algorithm{\vspace{.10in}\textbf{Algorithm: }}
\newcommand\correctness{\vspace{.10in}\textbf{Correctness: }}
\newcommand\runtime{\vspace{.10in}\textbf{Running time: }}
\pagestyle{fancyplain}
\lhead{\textbf{\NAME\ (\ANDREWID)}}
\chead{\textbf{Problem \HWNUM}}
\rhead{Strang., 18.06SC}
\begin{document}\raggedright
%Section A==============Change the values below to match your information==================
\newcommand\NAME{Haiying Cui}  % your name
\newcommand\ANDREWID{Christy}     % your andrew id
\newcommand\HWNUM{22}              % the homework number
%Section B==============Put your answers to the questions below here=======================

\title{Exercises on diagonalization and powers of A}
\maketitle
% no need to restate the problem --- the graders know which problem is which,
% but replacing "The First Problem" with a short phrase will help you remember
% which problem this is when you read over your homeworks to study.

\question{22.1}{}
In order to find the eigenvectors of \(A\), we first need to find the eiganvalues by solving:
$$det(A - \lambda I) = 0$$
$$det \begin{bmatrix} 4-\lambda \ 0 \\ 1 \ 2-\lambda \end{bmatrix} = det \begin{bmatrix} 4-\lambda \ 0 \\ 0 \ 2-lambda \end{bmatrix} = (4-\lambda)(2-\lambda) = 0$$
$$\lambda = 4, 2 $$
When \(\lambda = 4\), \((A-\lambda I)x = \begin{bmatrix} 0 \ 0 \\ 1 \ -2 \end{bmatrix} x = 0 \,\to\, x = \begin{bmatrix} 2 \\ 1\end{bmatrix}\).

When \(\lambda = 2\), \((A-\lambda I)x = \begin{bmatrix} 2 \ 0 \\ 1 \ 0 \end{bmatrix} x = 0 \,\to\, x = \begin{bmatrix} 0 \\ 1 \end{bmatrix}\).

Therefore, \(S\) has the columns of multiples of the two eiganvectors above.

The matrices that diagonalize \(A^{-1}\) is the same because \(A^{-1} = S\Lambda^{-1}S^{-1}\).

\question{22.2}{}
To find \(\Lambda\) and \(S\) of \(A\), we can solve:
$$det \begin{bmatrix} 0.6-\lambda \ 9 \\ 0.4 \ 0.1-\lambda \end{bmatrix} = det \begin{bmatrix} 0.6-\lambda \ 0.9 \\ 0 \ (0.1-\lambda)-\frac{0.36}{0.6-\lambda}\end{bmatrix}= \frac{0.7 \pm \sqrt{1.69}}{2}$$
$$\lambda = 1, -0.3$$
When \(\lambda = 1\), \((A-\lambda I)x = \begin{bmatrix} -0.4 \ 0.9 \\ 0.4 \ -0.9 \end{bmatrix} x = 0 \,\to\, x = \begin{bmatrix} 0.9 \\ 0.4\end{bmatrix}\).

When \(\lambda = -0.3\), \((A-\lambda I)x = \begin{bmatrix} 0.9 \ 0.9 \\ 0.4 \ 0.4 \end{bmatrix} x = 0 \,\to\, x = \begin{bmatrix} 1 \\ -1 \end{bmatrix}\).

Therefore, \(\Lambda = \begin{bmatrix} 1 \ 0 \\ 0 \ -0.3 \end{bmatrix}\) and \(S = \begin{bmatrix} 9 \ 1 \\ 4 \ -1 \end{bmatrix}\).

As \(\Lambda^k \,\to\, \infty, \Lambda \,\to\, \begin{bmatrix} 1 \ 0 \\ 0 \ 0 \end{bmatrix}\) and \(S\Lambda^{\infty}S^{-1} \,\to\, \begin{bmatrix} 9 \ 1 \\ 4 \ -1 \end{bmatrix}\begin{bmatrix} 1 \ 0 \\ 0 \ 0 \end{bmatrix}(-\frac{1}{13})\begin{bmatrix} -1 \ -1 \\ -4 \ 9 \end{bmatrix} = \frac{1}{13}\begin{bmatrix} 9 \ 9 \\ 4 \ 4 \end{bmatrix}\).
The columns of this matrix are steady state vectors. 
\end{document}