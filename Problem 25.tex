%=======================02-713 LaTeX template, following the 15-210 template==================
%
% You don't need to use LaTeX or this template, but you must turn your homework in as
% a typeset PDF somehow.
%
% How to use:
%    1. Update your information in section "A" below
%    2. Write your answers in section "B" below. Precede answers for all 
%       parts of a question with the command "\question{n}{desc}" where n is
%       the question number and "desc" is a short, one-line description of 
%       the problem. There is no need to restate the problem.
%    3. If a question has multiple parts, precede the answer to part x with the
%       command "\part{x}".
%    4. If a problem asks you to design an algorithm, use the commands
%       \algorithm, \correctness, \runtime to precede your discussion of the 
%       description of the algorithm, its correctness, and its running time, respectively.
%    5. You can include graphics by using the command \includegraphics{FILENAME}
%
\documentclass[11pt]{article}
\usepackage{amsmath,amssymb,amsthm}
\usepackage{graphicx}
\usepackage[margin=1in]{geometry}
\usepackage{fancyhdr}
\setlength{\parindent}{0pt}
\setlength{\parskip}{5pt plus 1pt}
\setlength{\headheight}{13.6pt}
\newcommand\question[2]{\vspace{.25in}\hrule\textbf{#1 #2}\vspace{.5em}\hrule\vspace{.10in}}
\renewcommand\part[1]{\vspace{.10in}\textbf{(#1)}}
\newcommand\algorithm{\vspace{.10in}\textbf{Algorithm: }}
\newcommand\correctness{\vspace{.10in}\textbf{Correctness: }}
\newcommand\runtime{\vspace{.10in}\textbf{Running time: }}
\pagestyle{fancyplain}
\lhead{\textbf{\NAME\ (\ANDREWID)}}
\chead{\textbf{Problem \HWNUM}}
\rhead{Strang., 18.06SC}
\begin{document}\raggedright
%Section A==============Change the values below to match your information==================
\newcommand\NAME{Haiying Cui}  % your name
\newcommand\ANDREWID{Christy}     % your andrew id
\newcommand\HWNUM{25}              % the homework number
%Section B==============Put your answers to the questions below here=======================

\title{symmetric matrices and positive definiteness}
\maketitle
% no need to restate the problem --- the graders know which problem is which,
% but replacing "The First Problem" with a short phrase will help you remember
% which problem this is when you read over your homeworks to study.

\question{25.1}{}
Writing out the first step of the proof:
$$Ax = \lambda x$$
left-hand side:$$\begin{bmatrix} 0 \ -1 \\ 1 \ 0 \end{bmatrix}\begin{bmatrix} i \\ 1 \end{bmatrix} = \begin{bmatrix} -1 \\ i \end{bmatrix}$$
right-hand side:
$$\begin{bmatrix} i^2 \\ i \end{bmatrix} = \begin{bmatrix} -1 \\ i \end{bmatrix} \checkmark$$

Writing out the second step:
$$x^TAx = \begin{bmatrix} i \ 1 \end{bmatrix}\begin{bmatrix} 0 \ -1 \\ 1 \\ 0 \end{bmatrix}\begin{bmatrix} i \\ 1 \end{bmatrix} = \begin{bmatrix} i \ 1 \end{bmatrix}\begin{bmatrix} -1 \\ i \end{bmatrix} = -i+i = 0$$
$$\lambda x^Tx = i\begin{bmatrix} i \ 1 \end{bmatrix}\begin{bmatrix} i \\ 1 \end{bmatrix} = i(i^2 + 1) =0 \checkmark $$

The third step, \(\lambda = \frac{x^TAx}{x^Tx}\), is under the assumption that \(x^Tx\) does not equal zero, and in the example above the product is zero. That's why the proof is false.

\question{25.2}{}
\part{a} If \(A\) and \(B\) are a positive definite symmetric matrices, for example, 

\(A = \begin{bmatrix} 1 \ 3 \\ 3 \ 10 \end{bmatrix}\) and \(B = \begin{bmatrix} 1 \ 2 \\ 2 \ 5 \end{bmatrix}\).
Then \(AB = \begin{bmatrix} 7 \ 23 \\ 15 \ 76 \end{bmatrix}\) which is not symmetric. So, \(A\) is not a group.

\part{b} If \(Q\) is orthogonal, \((Q^{-1})^TQ^{-1} = QQ^{-1} = I\). If \(P\) and \(Q\) are orthogonal matrices, then \((PQ)^TPQ = Q^TP^TPQ = I\). The results are both still orthogonal; \(Q\) is a group.

\part{c} 
$$(e^{tA})^{-1} = e^{-tA}$$
$$e^{mA}e^{nA} = e^{(m+n)A}$$ 
The exponential is a group.

\part{d} 
$$D^{-1} = \frac{1}{det(D)}D = D$$
$$det(AB) = det(A)det(B) = 1$$
Both matrices still have determinant 1. \(D\) is a group.
\end{document}