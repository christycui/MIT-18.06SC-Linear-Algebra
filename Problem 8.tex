%=======================02-713 LaTeX template, following the 15-210 template==================
%
% You don't need to use LaTeX or this template, but you must turn your homework in as
% a typeset PDF somehow.
%
% How to use:
%    1. Update your information in section "A" below
%    2. Write your answers in section "B" below. Precede answers for all 
%       parts of a question with the command "\question{n}{desc}" where n is
%       the question number and "desc" is a short, one-line description of 
%       the problem. There is no need to restate the problem.
%    3. If a question has multiple parts, precede the answer to part x with the
%       command "\part{x}".
%    4. If a problem asks you to design an algorithm, use the commands
%       \algorithm, \correctness, \runtime to precede your discussion of the 
%       description of the algorithm, its correctness, and its running time, respectively.
%    5. You can include graphics by using the command \includegraphics{FILENAME}
%
\documentclass[11pt]{article}
\usepackage{amsmath,amssymb,amsthm}
\usepackage{graphicx}
\usepackage[margin=1in]{geometry}
\usepackage{fancyhdr}
\setlength{\parindent}{0pt}
\setlength{\parskip}{5pt plus 1pt}
\setlength{\headheight}{13.6pt}
\newcommand\question[2]{\vspace{.25in}\hrule\textbf{#1 #2}\vspace{.5em}\hrule\vspace{.10in}}
\renewcommand\part[1]{\vspace{.10in}\textbf{(#1)}}
\newcommand\algorithm{\vspace{.10in}\textbf{Algorithm: }}
\newcommand\correctness{\vspace{.10in}\textbf{Correctness: }}
\newcommand\runtime{\vspace{.10in}\textbf{Running time: }}
\pagestyle{fancyplain}
\lhead{\textbf{\NAME\ (\ANDREWID)}}
\chead{\textbf{Problem \HWNUM}}
\rhead{Strang., 18.06SC}
\begin{document}\raggedright
%Section A==============Change the values below to match your information==================
\newcommand\NAME{Haiying Cui}  % your name
\newcommand\ANDREWID{Christy}     % your andrew id
\newcommand\HWNUM{8}              % the homework number
%Section B==============Put your answers to the questions below here=======================

\title{Exercises on solving Ax=0 and row reduced form R}
\maketitle
% no need to restate the problem --- the graders know which problem is which,
% but replacing "The First Problem" with a short phrase will help you remember
% which problem this is when you read over your homeworks to study.

\question{8.1}{}
\part{a} The coefficient of the particular solution is 1.

\part{b} The complete solution includes solution(s) to \(Ax = 0\). The complete solution \(x_p + cx_n \) can be more than one.

\part{c} \(Ax = 0\) always has the zero vector as a solution.

\question{8.2}{}
\(\begin{bmatrix}\begin{array}{c|c} U & 0 \end{array}\end{bmatrix}\,\to\,\begin{bmatrix}\begin{array}{ccc|c} 1&2&3&0 \\ 0&0&4&0 \end{array}\end{bmatrix}\,\to\,\begin{bmatrix}\begin{array}{ccc|c} 1&2&3&0 \\ 0&0&1&0 \end{array}\end{bmatrix}\,\to\,\begin{bmatrix}\begin{array}{ccc|c} 1&2&0&0 \\ 0&0&1&0 \end{array}\end{bmatrix}\)

Therefore, we have \(R = \begin{bmatrix} 1 \ 2 \ 0 \\ 0 \ 0 \ 1 \end{bmatrix}\). To solve for \(Rx = 0\), \(x = \begin{bmatrix} -2 \\ 1 \\ 0 \end{bmatrix}\) when \(x_2 = 1\).

\(\begin{bmatrix}\begin{array}{c|c} U & c \end{array}\end{bmatrix}\,\to\,\begin{bmatrix}\begin{array}{ccc|c} 1&2&3&5 \\ 0&0&4&8 \end{array}\end{bmatrix}\,\to\,\begin{bmatrix}\begin{array}{ccc|c} 1&2&3&5 \\ 0&0&1&2 \end{array}\end{bmatrix}\,\to\,\begin{bmatrix}\begin{array}{ccc|c} 1&2&0&-1 \\ 0&0&1&2 \end{array}\end{bmatrix}\)

Therefore, we have \(R = \begin{bmatrix} 1 \ 2 \ 0 \\ 0 \ 0 \ 1 \end{bmatrix}\) and \(d = \begin{bmatrix} -1 \\ 2 \end{bmatrix}\). To solve for \(Rx = d\), \(x = \begin{bmatrix} -3 \\ 1 \\ 2 \end{bmatrix}\) when \(x_2 = 1\).

\question{8.3}{}
If \(A = C\), then \(Ay\) has to equal \(Cy\) for all vectors \(y\). Since \(Ax = b\) has solutions to every \(b\), then y has to be one of the solutions. Similarly, it follows that \(y\) is also one of the solutions for \(Cx = b\). So, \(Ay = b = Cy\).

\end{document}
