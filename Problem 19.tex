%=======================02-713 LaTeX template, following the 15-210 template==================
%
% You don't need to use LaTeX or this template, but you must turn your homework in as
% a typeset PDF somehow.
%
% How to use:
%    1. Update your information in section "A" below
%    2. Write your answers in section "B" below. Precede answers for all 
%       parts of a question with the command "\question{n}{desc}" where n is
%       the question number and "desc" is a short, one-line description of 
%       the problem. There is no need to restate the problem.
%    3. If a question has multiple parts, precede the answer to part x with the
%       command "\part{x}".
%    4. If a problem asks you to design an algorithm, use the commands
%       \algorithm, \correctness, \runtime to precede your discussion of the 
%       description of the algorithm, its correctness, and its running time, respectively.
%    5. You can include graphics by using the command \includegraphics{FILENAME}
%
\documentclass[11pt]{article}
\usepackage{amsmath,amssymb,amsthm}
\usepackage{graphicx}
\usepackage[margin=1in]{geometry}
\usepackage{fancyhdr}
\setlength{\parindent}{0pt}
\setlength{\parskip}{5pt plus 1pt}
\setlength{\headheight}{13.6pt}
\newcommand\question[2]{\vspace{.25in}\hrule\textbf{#1 #2}\vspace{.5em}\hrule\vspace{.10in}}
\renewcommand\part[1]{\vspace{.10in}\textbf{(#1)}}
\newcommand\algorithm{\vspace{.10in}\textbf{Algorithm: }}
\newcommand\correctness{\vspace{.10in}\textbf{Correctness: }}
\newcommand\runtime{\vspace{.10in}\textbf{Running time: }}
\pagestyle{fancyplain}
\lhead{\textbf{\NAME\ (\ANDREWID)}}
\chead{\textbf{Problem \HWNUM}}
\rhead{Strang., 18.06SC}
\begin{document}\raggedright
%Section A==============Change the values below to match your information==================
\newcommand\NAME{Haiying Cui}  % your name
\newcommand\ANDREWID{Christy}     % your andrew id
\newcommand\HWNUM{19}              % the homework number
%Section B==============Put your answers to the questions below here=======================

\title{Exercises on determinant formulas and cofactors}
\maketitle
% no need to restate the problem --- the graders know which problem is which,
% but replacing "The First Problem" with a short phrase will help you remember
% which problem this is when you read over your homeworks to study.

\question{19.1}{}
If we switch row 1 and row 2, then row 2 and row 4 and lastly row 2 and row 3, we can go from \(A\) to \(I\). Since there are three row exchanges, the determinant of \(A\) is -1.

\question{19.2}{}
We can rewrite the determinants as:

det \(\begin{bmatrix} 1 \ 1 \ 1 \ 1 \\ 1 \ 2 \ 3 \ 4 \\ 1 \ 3 \ 6 \ 10 \\ 1 \ 4 \ 10 \ 20 \end{bmatrix}\) = det \(\begin{bmatrix} 2 \ 3 \ 4 \\ 3 \ 6 \ 10 \\ 4 \ 10 \ 20 \end{bmatrix}\) - det \(\begin{bmatrix} 1 \ 3 \ 4 \\ 1 \ 6 \ 10 \\ 1 \ 10 \ 20 \end{bmatrix}\) + det \(\begin{bmatrix} 1 \ 2 \ 4 \\ 1 \ 3 \ 10 \\ 1 \ 4 \ 20 \end{bmatrix}\) - det \(\begin{bmatrix} 1 \ 2 \ 3 \\ 1 \ 3 \ 6 \\ 1 \ 4 \ 10 \end{bmatrix}\) = 1

det \(\begin{bmatrix} 1 \ 1 \ 1 \ 1 \\ 1 \ 2 \ 3 \ 4 \\ 1 \ 3 \ 6 \ 10 \\ 1 \ 4 \ 10 \ (20-1) \end{bmatrix}\) = det \(\begin{bmatrix} 2 \ 3 \ 4 \\ 3 \ 6 \ 10 \\ 4 \ 10 \ (20-1) \end{bmatrix}\) - det \(\begin{bmatrix} 1 \ 3 \ 4 \\ 1 \ 6 \ 10 \\ 1 \ 10 \ (20-1) \end{bmatrix}\) + det \(\begin{bmatrix} 1 \ 2 \ 4 \\ 1 \ 3 \ 10 \\ 1 \ 4 \ (20-1) \end{bmatrix}\) - det \(\begin{bmatrix} 1 \ 2 \ 3 \\ 1 \ 3 \ 6 \\ 1 \ 4 \ 10 \end{bmatrix}\)

We can observe the difference of the two determinants:

det \(\begin{bmatrix} 1 \ 1 \ 1 \ 1 \\ 1 \ 2 \ 3 \ 4 \\ 1 \ 3 \ 6 \ 10 \\ 1 \ 4 \ 10 \ 20 \end{bmatrix}\) - det \(\begin{bmatrix} 1 \ 1 \ 1 \ 1 \\ 1 \ 2 \ 3 \ 4 \\ 1 \ 3 \ 6 \ 10 \\ 1 \ 4 \ 10 \ 19 \end{bmatrix}\) = \(2\times6 - 3\times3 - (1\times6 - 3\times1) + (1\times3 - 2\times1) = 12 - 9 - (6 - 3) + 1 = 1\)

Therefore, the determinant of \(\begin{bmatrix} 1 \ 1 \ 1 \ 1 \\ 1 \ 2 \ 3 \ 4 \\ 1 \ 3 \ 6 \ 10 \\ 1 \ 4 \ 10 \ 19 \end{bmatrix}\) is 1 - 1 = 0.
\end{document}