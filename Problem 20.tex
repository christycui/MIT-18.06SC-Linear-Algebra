%=======================02-713 LaTeX template, following the 15-210 template==================
%
% You don't need to use LaTeX or this template, but you must turn your homework in as
% a typeset PDF somehow.
%
% How to use:
%    1. Update your information in section "A" below
%    2. Write your answers in section "B" below. Precede answers for all 
%       parts of a question with the command "\question{n}{desc}" where n is
%       the question number and "desc" is a short, one-line description of 
%       the problem. There is no need to restate the problem.
%    3. If a question has multiple parts, precede the answer to part x with the
%       command "\part{x}".
%    4. If a problem asks you to design an algorithm, use the commands
%       \algorithm, \correctness, \runtime to precede your discussion of the 
%       description of the algorithm, its correctness, and its running time, respectively.
%    5. You can include graphics by using the command \includegraphics{FILENAME}
%
\documentclass[11pt]{article}
\usepackage{amsmath,amssymb,amsthm}
\usepackage{graphicx}
\usepackage[margin=1in]{geometry}
\usepackage{fancyhdr}
\setlength{\parindent}{0pt}
\setlength{\parskip}{5pt plus 1pt}
\setlength{\headheight}{13.6pt}
\newcommand\question[2]{\vspace{.25in}\hrule\textbf{#1 #2}\vspace{.5em}\hrule\vspace{.10in}}
\renewcommand\part[1]{\vspace{.10in}\textbf{(#1)}}
\newcommand\algorithm{\vspace{.10in}\textbf{Algorithm: }}
\newcommand\correctness{\vspace{.10in}\textbf{Correctness: }}
\newcommand\runtime{\vspace{.10in}\textbf{Running time: }}
\pagestyle{fancyplain}
\lhead{\textbf{\NAME\ (\ANDREWID)}}
\chead{\textbf{Problem \HWNUM}}
\rhead{Strang., 18.06SC}
\begin{document}\raggedright
%Section A==============Change the values below to match your information==================
\newcommand\NAME{Haiying Cui}  % your name
\newcommand\ANDREWID{Christy}     % your andrew id
\newcommand\HWNUM{20}              % the homework number
%Section B==============Put your answers to the questions below here=======================

\title{Exercises on Cramer's rule, inverse matrix, and volume}
\maketitle
% no need to restate the problem --- the graders know which problem is which,
% but replacing "The First Problem" with a short phrase will help you remember
% which problem this is when you read over your homeworks to study.

\question{20.1}{}
$$C = \begin{bmatrix} 6 \ -3 \ 0 \\ 3 \ 1 \ -1 \\ -6 \ 2 \ 1 \end{bmatrix}$$
$$AC^T = \begin{bmatrix} 1 \ 1 \ 4 \\ 1 \ 2 \ 2 \\ 1 \ 2 \ 5 \end{bmatrix}\begin{bmatrix} 6 \ 3 \ -6 \\ -3 \ 1 \ 2 \\ 0 \ -1 \ 1 \end{bmatrix} = \begin{bmatrix} 3 \ 0 \ 0 \\ 0 \ 3 \ 0 \\ 0 \ 0 \ 3 \end{bmatrix} = 3I = det(A)I$$
$$det(A) = 3$$
Changing \(A_{13}\) to 100 wouldn't change the determinant because on the matrix factor the position of that entry is 0, so it doesn't affect the determinant.

\question{20.2}{}
The matrix of partial derivatives is:
$$\begin{bmatrix} sin\phi cos\theta \ \rho cos\theta cos \phi \ -\rho sin \ theta sin \ phi \\ sin \phi sin \theta \ \rho sin \theta cos \phi \ \rho sin \phi cos \theta \\ cos \phi \ -\rho sin \phi \ 0\end{bmatrix}$$
Using cofactor formula to find the determinant of the above matrix:
\(J = sin\phi cos\theta (\rho^2 sin^2\phi cos\theta) - \rho cos\theta cos\phi(-\rho sin\phi cos\phi cos\theta) -
 (-\rho sin\theta sin\phi)(-\rho sin^2\phi sin\theta - \rho sin\theta cos^2\phi) = \rho^2 sin^3\phi cos^2\theta + \rho^2 sin\phi cos^2\phi cos^2\theta + \rho^2 sin^3\phi sin^2 \theta + \rho^2 sin^2\theta sin\phi cos^2\phi = \rho^2 sin^3\phi + \rho^2 sin\phi cos^2\phi = \rho^2 sin\phi\)
\end{document}