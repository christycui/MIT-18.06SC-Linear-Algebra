%=======================02-713 LaTeX template, following the 15-210 template==================
%
% You don't need to use LaTeX or this template, but you must turn your homework in as
% a typeset PDF somehow.
%
% How to use:
%    1. Update your information in section "A" below
%    2. Write your answers in section "B" below. Precede answers for all 
%       parts of a question with the command "\question{n}{desc}" where n is
%       the question number and "desc" is a short, one-line description of 
%       the problem. There is no need to restate the problem.
%    3. If a question has multiple parts, precede the answer to part x with the
%       command "\part{x}".
%    4. If a problem asks you to design an algorithm, use the commands
%       \algorithm, \correctness, \runtime to precede your discussion of the 
%       description of the algorithm, its correctness, and its running time, respectively.
%    5. You can include graphics by using the command \includegraphics{FILENAME}
%
\documentclass[11pt]{article}
\usepackage{amsmath,amssymb,amsthm}
\usepackage{graphicx}
\usepackage[margin=1in]{geometry}
\usepackage{fancyhdr}
\setlength{\parindent}{0pt}
\setlength{\parskip}{5pt plus 1pt}
\setlength{\headheight}{13.6pt}
\newcommand\question[2]{\vspace{.25in}\hrule\textbf{#1 #2}\vspace{.5em}\hrule\vspace{.10in}}
\renewcommand\part[1]{\vspace{.10in}\textbf{(#1)}}
\newcommand\algorithm{\vspace{.10in}\textbf{Algorithm: }}
\newcommand\correctness{\vspace{.10in}\textbf{Correctness: }}
\newcommand\runtime{\vspace{.10in}\textbf{Running time: }}
\pagestyle{fancyplain}
\lhead{\textbf{\NAME\ (\ANDREWID)}}
\chead{\textbf{Problem \HWNUM}}
\rhead{Strang., 18.06SC}
\begin{document}\raggedright
%Section A==============Change the values below to match your information==================
\newcommand\NAME{Haiying Cui}  % your name
\newcommand\ANDREWID{Christy}     % your andrew id
\newcommand\HWNUM{16}              % the homework number
%Section B==============Put your answers to the questions below here=======================

\title{Exercises on projection matrices and least squares}
\maketitle
% no need to restate the problem --- the graders know which problem is which,
% but replacing "The First Problem" with a short phrase will help you remember
% which problem this is when you read over your homeworks to study.

\question{16.1}{}
We can rewrite the problem as \(\begin{bmatrix} 1 \ -1 \\ 1 \ 1 \\ 1 \ 2 \end{bmatrix}x = \begin{bmatrix}7 \\ 7 \\ 21 \end{bmatrix}\). Multiply each side by \(A^T\), we want to find \(\hat{x}\) that satisfies \(A^TA\hat{x} = A^Tb\).
$$A^TA = \begin{bmatrix} 1 \ 1 \ 1 \\ -1 \ 1 \ 2 \end{bmatrix}\begin{bmatrix} 1 \ -1 \\ 1 \ 1 \\ 1 \ 2 \end{bmatrix} = \begin{bmatrix} 3 \ 2 \\ 2 \ 6 \end{bmatrix}$$
$$A^Tb = \begin{bmatrix} 1 \ 1 \ 1 \\ -1 \ 1 \ 2 \end{bmatrix}\begin{bmatrix} 7 \\ 7 \\ 21 \end{bmatrix} = \begin{bmatrix} 35 \\ 42 \end{bmatrix}$$
$$\begin{bmatrix} 3 \ 2 \\ 2 \ 6 \end{bmatrix}\begin{bmatrix} C \\ D \end{bmatrix} = \begin{bmatrix} 35 \\ 42 \end{bmatrix}$$

We can solve the above and get \(\hat{x} = \begin{bmatrix} 9 \\ 4 \end{bmatrix}\).

\question{16.2}{}
$$p = A\hat{x} = \begin{bmatrix} 1 \ -1 \\ 1 \ 1 \\ 1 \ 2 \end{bmatrix}\begin{bmatrix} 9 \\ 4 \end{bmatrix} = \begin{bmatrix} 5 \\ 13 \\ 17 \end{bmatrix}$$
$$e = b - p = \begin{bmatrix} 2 \\ -6 \\ 4 \end{bmatrix}$$
$$Pe = P(b - p) = Pb - Pp = p - p = 0$$

\question{16.3}{}
When \(b = \begin{bmatrix} 2 \\ -6 \\ 4 \end{bmatrix}\), \(b\) is perpendicular to the column space of \(A\).

\question{16.4}{}
\(hat{x}\) remains the same as before. The error \(e = 0\) because \(b\) is in the column space of \(A\).

\question{16.5}{}
The error vector \(e\) is in the left nullspace of \(A\). \(p\) is in the column space of \(A\). \(\hat{x}\) is in the row space of \(A\). The nullspace of \(A\) is the zero vector.

\question{16.6}{}
In an attempt to solve \(\begin{bmatrix} 1 \ -2 \\ 1 \ -1 \\ 1 \ 0 \\ 1 \ 1 \\ 1 \ 2 \end{bmatrix}x = \begin{bmatrix} 4 \\ 2 \\ -1 \\ 0 \\ 0 \end{bmatrix}\), we multiply both sides by \(A^T\).
$$A^TA = \begin{bmatrix} -5 \ 0 \\ 0 \ 10 \end{bmatrix}$$
$$A^Tb = \begin{bmatrix} 5 \\ -10 \end{bmatrix}$$
$$\begin{bmatrix} -5 \ 0 \\ 0 \ 10 \end{bmatrix}\begin{bmatrix} C \\ D \end{bmatrix} = \begin{bmatrix} 5 \\ -10 \end{bmatrix}$$
Solving the above, we get \(\hat{x} = \begin{bmatrix} 1 \\ -1 \end{bmatrix}\). Therefore, the best line is \(y = 1 - t\).
\end{document}