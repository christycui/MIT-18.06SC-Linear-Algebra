%=======================02-713 LaTeX template, following the 15-210 template==================
%
% You don't need to use LaTeX or this template, but you must turn your homework in as
% a typeset PDF somehow.
%
% How to use:
%    1. Update your information in section "A" below
%    2. Write your answers in section "B" below. Precede answers for all 
%       parts of a question with the command "\question{n}{desc}" where n is
%       the question number and "desc" is a short, one-line description of 
%       the problem. There is no need to restate the problem.
%    3. If a question has multiple parts, precede the answer to part x with the
%       command "\part{x}".
%    4. If a problem asks you to design an algorithm, use the commands
%       \algorithm, \correctness, \runtime to precede your discussion of the 
%       description of the algorithm, its correctness, and its running time, respectively.
%    5. You can include graphics by using the command \includegraphics{FILENAME}
%
\documentclass[11pt]{article}
\usepackage{amsmath,amssymb,amsthm}
\usepackage{graphicx}
\usepackage[margin=1in]{geometry}
\usepackage{fancyhdr}
\setlength{\parindent}{0pt}
\setlength{\parskip}{5pt plus 1pt}
\setlength{\headheight}{13.6pt}
\newcommand\question[2]{\vspace{.25in}\hrule\textbf{#1 #2}\vspace{.5em}\hrule\vspace{.10in}}
\renewcommand\part[1]{\vspace{.10in}\textbf{(#1)}}
\newcommand\algorithm{\vspace{.10in}\textbf{Algorithm: }}
\newcommand\correctness{\vspace{.10in}\textbf{Correctness: }}
\newcommand\runtime{\vspace{.10in}\textbf{Running time: }}
\pagestyle{fancyplain}
\lhead{\textbf{\NAME\ (\ANDREWID)}}
\chead{\textbf{Problem \HWNUM}}
\rhead{Strang., 18.06SC}
\begin{document}\raggedright
%Section A==============Change the values below to match your information==================
\newcommand\NAME{Haiying Cui}  % your name
\newcommand\ANDREWID{Christy}     % your andrew id
\newcommand\HWNUM{4}              % the homework number
%Section B==============Put your answers to the questions below here=======================

\title{Exercises on Factorization into A = LU}
\maketitle
% no need to restate the problem --- the graders know which problem is which,
% but replacing "The First Problem" with a short phrase will help you remember
% which problem this is when you read over your homeworks to study.

\question{4.1}{}
First, we go through elimination on matrix \(A\). To eliminate the first column:
$$\begin{bmatrix} 1 \ 3 \ 0 \\ 2 \ 4 \ 0 \\ 2 \ 0 \ 1 \end{bmatrix}\begin{bmatrix} 1 \ 0 \ 0 \\ -2 \ 1 \ 0 \\ -2 \ 0 \ 1 \end{bmatrix} = \begin{bmatrix} 1 \ 3 \ 0 \\ 0 \ -2 \ 0 \\ 0 \ -6 \ 1 \end{bmatrix}$$
To eliminate the second column:
$$\begin{bmatrix} 1 \ 3 \ 0 \\ 0 \ -2 \ 0 \\ 0 \ -6 \ 1 \end{bmatrix}\begin{bmatrix} 1 \ 0 \ 0 \\ 0 \ 1 \ 0 \\ 0 \ -3 \ 1 \end{bmatrix} = \begin{bmatrix} 1 \ 3 \ 0 \\ 0 \ -2 \ 0 \\ 0 \ 0 \ 1 \end{bmatrix}$$
To summarize the elimination process, we have \(E_2E_1A = U\) where \(E_1 = \begin{bmatrix} 1 \ 0 \ 0 \\ -2 \ 1 \ 0 \\ -2 \ 0 \ 1 \end{bmatrix}\) and \(E_2 = \begin{bmatrix} 1 \ 0 \ 0 \\ 0 \ 1 \ 0 \\ 0 \ -3 \ 1 \end{bmatrix}\).

We can rewrite the equation into \(A = E_1^{-1}E_2^{-1}U\) and \(A = LU\). Then it leads to 
$$L = E_1^{-1}E_2^{-1} = \begin{bmatrix} 1 \ 0 \ 0 \\ -2 \ 1 \ 0 \\ -2 \ 0 \ 1 \end{bmatrix}\begin{bmatrix} 1 \ 0 \ 0 \\ 0 \ 1 \ 0 \\ 0 \ -3 \ 1 \end{bmatrix} = \begin{bmatrix} 1 \ 0 \ 0 \\ 2 \ 1 \ 0 \\ 2 \ 3 \ 1 \end{bmatrix}$$

\question{4.2}{}
To eliminate the first column:
$$\begin{bmatrix} a \ a \ a \ a \\ a \ b \ b \ b \\ a \ b \ c \ c \\ a \ b \ c \ d \end{bmatrix}\begin{bmatrix} 1 \ 0 \ 0 \ 0 \\ -1 \ 1 \ 0 \ 0 \\ -1 \ 0 \ 1 \ 0 \\ -1 \ 0 \ 0 \ 1\end{bmatrix} = \begin{bmatrix} a \ a \ a \ a \\ 0 \ b-a \ b-a \ b-a \\ 0 \ b-a \ c-a \ c-a \\ 0 \ b-a \ c-a \ d-a \end{bmatrix}$$
To eliminate the second column:
$$\begin{bmatrix} a \ a \ a \ a \\ 0 \ b-a \ b-a \ b-a \\ 0 \ b-a \ c-a \ c-a \\ 0 \ b-a \ c-a \ d-a \end{bmatrix}\begin{bmatrix} 1 \ 0 \ 0 \ 0 \\ 0 \ 1 \ 0 \ 0 \\ 0 \ -1 \ 1 \ 0 \\ 0 \ -1 \ 0 \ 1 \end{bmatrix} = \begin{bmatrix} a \ a \ a \ a \\ 0 \ b-a \ b-a \ bia \\ 0 \ 0 \ c-b \ c-b \\ 0 \ 0 \ c-b \ d-b \end{bmatrix}$$
To eliminate the third column:
$$\begin{bmatrix} a \ a \ a \ a \\ 0 \ b-a \ b-a \ bia \\ 0 \ 0 \ c-b \ c-b \\ 0 \ 0 \ c-b \ d-b \end{bmatrix}\begin{bmatrix} 1 \ 0 \ 0 \ 0 \\ 0 \ 1 \ 0 \ 0 \\ 0 \ 0 \ 1 \ 0 \\ 0 \ 0 \ -1 \ 1 \end{bmatrix} = \begin{bmatrix} a \ a \ a \ a \\ 0 \ b-a \ b-a \ bia \\ 0 \ 0 \ c-b \ c-b \\ 0 \ 0 \ 0 \ d-c \end{bmatrix}$$
To summarize, we have \(E_3E_2E_1A = U\), so \(A = E_1^{-1}E_2^{-1}E_3^{-1}U\). Therefore,
$$L = E_1^{-1}E_2^{-1}E_3^{-1} = \begin{bmatrix} 1 \ 0 \ 0 \ 0 \\ 1 \ 1 \ 0 \ 0 \\ 1 \ 1 \ 1 \ 0 \\ 1 \ 1 \ 1 \ 1\end{bmatrix}$$ 

The four conditions to get \(A = LU\) with four pivots are: \(a\neq0\), \(a\neq b\), \(b \neq c\) and \(c \neq d\).
\end{document}
